\documentclass[10pt,a4paper,ragged2e]{altacv}
\geometry{left=2cm,right=10cm,marginparwidth=6.8cm,marginparsep=1.2cm,top=1.25cm,bottom=1.25cm}
\usepackage{hyperref}
\ifxetexorluatex
  \setmainfont{Carlito}
\else
  \usepackage[utf8]{inputenc}
  \usepackage[T1]{fontenc}
  \usepackage[default]{lato}
\fi
\definecolor{VividPurple}{HTML}{000000}
\definecolor{SlateGrey}{HTML}{2E2E2E}
\definecolor{LightGrey}{HTML}{2E2E2E}
\colorlet{heading}{VividPurple}
\colorlet{accent}{VividPurple}
\colorlet{emphasis}{SlateGrey}
\colorlet{body}{LightGrey}

\renewcommand{\itemmarker}{{\small\textbullet}}
\renewcommand{\ratingmarker}{\faCircle}
\addbibresource{sample.bib}

\begin{document}
\name{Hugo Persson}
\tagline{MSc Computer Science and Engineering Student}
% Cropped to square from https://en.wikipedia.org/wiki/Marissa_Mayer#/media/File:Marissa_Mayer_May_2014_(cropped).jpg, CC-BY 2.0
%\photo{3.3cm}{profile.jpg}
\personalinfo{%
  % Not all of these are required!
  % You can add your own with \printinfo{symbol}{detail}
  \email{\href{mailto:hugo.e.persson@gmail.com}{hugo.e.persson@gmail.com}}
   \phone{\href{tel:+46722406005}{+46722406005}}
%  \mailaddress{Address, Street, 00000 County}
\newline
\smallskip
  \location{Lund - Sweden}
  %\homepage{\href{https://hugopersson.com}{hugopersson.com}}
  \github{\href{https://github.com/Hugo-Persson}{github.com/Hugo-Persson}}
%  \twitter{@marissamayer}
  \linkedin{ \href{https://www.linkedin.com/in/hugopersson7}{linkedin.com/in/hugopersson7} }
%   \orcid{orcid.org/0000-0000-0000-0000} % Obviously making this up too. If you want to use this field (and also other academicons symbols), add "academicons" option to \documentclass{altacv}
}

%% Make the header extend all the way to the right, if you want.
\begin{fullwidth}
\makecvheader
\end{fullwidth}

%% Depending on your tastes, you may want to make fonts of itemize environments slightly smaller
\AtBeginEnvironment{itemize}{\small}

%% Provide the file name containing the sidebar contents as an optional parameter to \cvsection.
%% You can always just use \marginpar{...} if you do
%% not need to align the top of the contents to any
%% \cvsection title in the "main" bar.
\cvsection[page1sidebar]{Experience}

\cvevent{App developer}{Combain Mobile}{Jan 2024 -- Present}{}
\begin{itemize}
\item Maintained Combain’s mobile apps across various programming languages, quickly adapting to different project structures. Technologies used include Jetpack Compose, Flutter, SwiftUI, and Objective-C.
\item Developed a local-first version of Combain’s positioning solutions for Android and iOS, employing ANN models with ONNX Runtime for on-device positioning using Wi-Fi, Bluetooth, and GPS.
\end{itemize}

\divider

\cvevent{Self-employed}{Evercode AB}{Aug 2022 -- Present}{}
\begin{itemize}
\item Founded and operated an IT consulting business, gaining valuable insights into business management, including employee costs, taxation, and regulations.
\item Secured and managed multiple consulting contracts, handling negotiations and preparing price quotes.
\end{itemize}

\divider

\cvevent{Fullstack developer}{Recruto}{Sep 2020 -- Jan 2024}{}
\begin{itemize}
\item Maintained Recruto’s web app using PHP, TypeScript, Node.js, MySQL, and Rust.
\item Enhanced system stability with on-premise Grafana, Loki, Prometheus, and Sentry.
\item Achieved 77x faster bulk email send-outs by implementing a RabbitMQ queue system.
\item Built a scalable notification system (Email, SMS, Web Push, mobile push) using Rust.
\item Developed and maintained the Recruto mobile app for iOS and Android in C\# (Xamarin Forms).
\end{itemize}

\divider

\cvevent{Teaching Assistant}{LTH}{Fall of 2022 \& 2023}{}
\begin{itemize}
\item Introductory Course in Programming 
\item Secondary Course in Programming
\end{itemize}

\divider

\cvevent{Event Coordinator}{Arkad}{Fall of 2023}{}
\begin{itemize}
\item Assisted in organizing the Arkad career fair, coordinating logistics and company engagement.
\item Managed communications with companies, coordinated food deliveries, and led a team of volunteers.
\end{itemize}


% \cvevent{Programmeringsolympiaden}{}{Spring 2019}{}
% \begin{itemize}
% \item Finalist in Competitive Programming competition for High School students
% \end{itemize}


% \cvevent{Product Engineer}{Google}{23 June 1999 -- 2001}{Palo Alto, CA}

% \begin{itemize}
% \item Joined the company as employe \#20 and female employee \#1
% \item Developed targeted advertisement in order to use user's search queries and show them related ads
% \end{itemize}

%\cvsection{A Day of My Life}

% Adapted from @Jake's answer from http://tex.stackexchange.com/a/82729/226
% \wheelchart{outer radius}{inner radius}{
% comma-separated list of value/text width/color/detail}
% Some ad-hoc tweaking to adjust the labels so that they don't overlap
% \wheelchart{1.5cm}{0.5cm}{%
%   10/10em/accent!30/Sleeping \& dreaming about work,
%   25/9em/accent!60/Public resolving issues with Yahoo!\ investors,
%   5/13em/accent!10/\footnotesize\\[1ex]New York \& San Francisco Ballet Jawbone board member,
%   20/15em/accent!40/Spending time with family,
%   5/8em/accent!20/\footnotesize Business development for Yahoo!\ after the Verizon acquisition,
%   30/9em/accent/Showing Yahoo!\ employees that their work has meaning,
%   5/8em/accent!20/Baking cupcakes
% }

\clearpage

% \cvsection[page2sidebar]{Publications}

\nocite{*}

% \printbibliography[heading=pubtype,title={\printinfo{\faBook}{Books}},type=book]

% \divider

% \printbibliography[heading=pubtype,title={\printinfo{\faFileTextO}{Journal Articles}}, type=article]

% \divider

% \printbibliography[heading=pubtype,title={\printinfo{\faGroup}{Conference Proceedings}},type=inproceedings]

% %% If the NEXT page doesn't start with a \cvsection but you'd
% %% still like to add a sidebar, then use this command on THIS
% %% page to add it. The optional argument lets you pull up the
% %% sidebar a bit so that it looks aligned with the top of the
% %% main column.
% % \addnextpagesidebar[-1ex]{page3sidebar}


\end{document}
